\documentclass{article}
\usepackage{graphicx}
\title{Setting Up Windows Subsystem for Linux}
\author{Garrett Ailts}
\date{\today}

\begin{document}


\maketitle

\section{What is Windows Subsystem for Linux (WSL)?}
Windows Subsystem for Linux is a compatibility layer that has been developed for Windows 10. It allows the user to run Linux binary executable and allows access to a GNU user space. What this means is you can work with a Linux style terminal and uses Linux commands and tools like git, apt-get, graphical applications, programming language interpreters like python or ruby, and more. In this tutorial, I'll explain how to set up Linux Subsystem for Windows, and how to use git commands and github in an Ubuntu terminal.

\section{Enabling Developer Mode and WSL}
In order to use WSL on your windows computer, you first have to enable developer mode and turn on WSL as a windows feature. Developer mode will allow you to more easily test apps and use the Ubuntu bash shell once you enable WSL. To enable developer mode go to settings -> update and security ->  for developers. Click on the open circle next to developer mode. Wait for windows to complete the action. Once developer mode is on, you can turn on WSL. To activate WSL, go to control panel -> programs , and click on turn windows features on or off. A list of windows features will pop up with check boxes next to them. Scroll down to the bottom and select the box next "Windows Subsystem for Linux." After windows is done activating WSL, it will prompt you to restart your computer, select ok.

\section{Installing and Setting Up Ubuntu}
Now that you have WSL enabled, you need to download a special image of the Ubuntu OS from the Windows store. Once the download is completed, type Ubuntu into your windows search bar, the bash command for starting the bash shell or the Ubuntu application should pop up, press enter. This will cause the bash shell to start up and begin setting up Ubuntu. After the setup finishes, the terminal will prompt you to create a Unix username and password. Keep in mind that when you type the password, nothing you type will actually show up in the terminal window, though Ubuntu will see it. Ubuntu does this for security. Once you've set up your account, you are now ready to start using WSL.

\section{Navigating the Terminal}
When the bash terminal starts up, the default starting location is a virtual desktop created by WSL. In order to access all the files on your hardrives (internal and external), you'll need to use the cd command to back up two directories. To back out of a directory, the command is \emph{cd ../../}. The two dots followed by a forward slash mean move up one directory or "back out" of your current directory level. Once you type any command in the bash terminal, you simply need to press enter to execute it. In order to enter a directory, the command is \emph{cd filepath} where \emph(filepath} is the path that describes where the directory you wish to travel to is. The filepath can be a relative path from the directory your in or the full filepath. The picture shown below shows how to navigate to your documents folder after starting the terminal. Notice that after we back up two directories, we use the \emph{dir} command to see the directories in our location. The directory \emph{mnt} is where all of the hardrives on your computer can be accessed. This includes your main hardrive partition, usually called \emph{c}, as well as any external drives you have plugged in to your computer. To back out to the top level directory, just use the command \emph{cd} with no arguments. This will work regardless of what directory you are in. Lastly, a helpful little tip for increasing your navigation speed in the terminal. Once you've typed enough letters of a directory name to the point where the name is unique, you can press \emph{tab} to auto complete. This means you don't have to worry about coming up with unique and excessively long names for folders, woohoo!

\section{Using Git in Bash Shell}
In this last section, we'll go over how to set up and use git/github in the bash terminal. First we'll create a directory, then we'll clone the repository into that directory, then we'll change a file, commit our changes, and push it to github. If all of that sounder Greek to you, please see Ryan Bowers \emph{Introduction to GitHub} presentation located here (replace with link). The first step is to find a good place to store all your local versions of github repositories. Personally, I prefer creating a directory called \emph{Repositories} in my Documents folder. Within this folder are directories for all the organizations and projects I am a part of. Within each project folder is where I store repositories of code related to that project. For the MURI team, I will have you set up this directory and clone a repository into it. First we'll navigate to the documents folder using the \emph{cd} command. Then we will use the command \emph{mkdir} to create a directory called repositories and cd into that folder. Now lets use the \emph{mkdir} command again to make a folder for the MURI project and cd into that folder. Once were in this folder, type the git command \emph{git clone} into the terminal. Don't press enter yet and leave a space after clone. Now lets go to github and go to the repositories main page. On the page, you'll see a green button that says \emph{clone or download}. Select this button, then click on the gray button to the right of the url in order to copy it. Go back to the terminal and right click to paste this url after \emph{git clone}. Press enter. Git will now create a copy of the repository on your local machine. Now cd into this repository. Once in the repository, you'll be able to use all the commands git has to offer. Before we do that though, there is something important we should do. If this is the first time you will be pushing to GitHub from which you copied the link to the repository, you will have to tell git who you are. To do this, you will need to run the following two commands. The first is \emph{git config --global user.name "Your github user name"}. The second is \emph{git config --global user.email "The email you used for your github account"}. Now, use the windows file explorer to open one of the files. Make a comment in the file and save it. Run the command \emph{git status}. It should show you the name of the file you changed in red. Next, use the command \emph{git add filename} to add the changes to the queue. Finally, make a commit using the command \emph{git commit -m "Message describing what you changed or what the commit is for."}. Now let's push our changes. Type the command \emph{git push origin gitTest}, where origin is the default name given to the remote, and gitTest is the branch name. You will be prompted to add your username and password, then git will push your changes and update the remote. Go check github and look for your new branch!


\end{document}
